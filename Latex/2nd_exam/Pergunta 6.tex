{$\space$\par}
\vspace{0.5cm}
\justifying
\section*{{\bfseries \LARGE Questão 6 -} {\bfseries \large Transforme as cores em magnitudes U, B, R e I, e junte essas magnitudes à V em uma nova dataframe. Aplique a técnica de componentes principais nesta dataframe formada apenas por magnitudes aparentes.}}

\vspace{0.3cm}

\begin{enumerate}
    \item Quantas componentes principais poderiam explicar apropriadamente a distribuição das estrelas nesse espaço de magnitudes?
        
    \item Interprete. O que deve significar fisicamente os dois primeiros componentes principais?

    \item Quais seriam as magnitudes de uma estrela hipotética que pudesse ser representada no espaço de componentes principais pelas coordenadas (0.2, 0.43, 0.6, 0.8, 0.4)?
\end{enumerate}
\vspace{0.8cm}

\textcolor{red}{O método de PCA rotaciona o espaço dos parâmetros de maneira que os primeiros eixos sejam aqueles onde há maior variância dos dados. Esse método é muito útil para reduzir a dimensionalidade do problema ao assumir que as componentes de menor variância são ruídos.}

\vspace{0.4cm}

\begin{lstlisting}
    catalog = read.table('/content/king5.tsv', sep='|', header=T)

    # Estimating magnitude
    Bmag = catalog$BV+catalog$Vmag
    Umag = catalog$UB+Bmag
    Rmag = -(catalog$VR-catalog$Vmag)
    Imag = -(catalog$VI-catalog$Vmag)
    
    new_df = data.frame(Umag = Umag, Bmag = Bmag, Vmag = catalog$Vmag, Rmag = Rmag,
     Imag = Imag
    )
    new_df = new_df[complete.cases(new_df),]
    
    # PCA
    pca = prcomp(new_df)
    summary(pca)
    
    ### output ### 
    Importance of components:
                              PC1     PC2     PC3     PC4     PC5
    Standard deviation     2.7876 0.42396 0.17145 0.04079 0.03337
    Proportion of Variance 0.9735 0.02252 0.00368 0.00021 0.00014
    Cumulative Proportion  0.9735 0.99597 0.99965 0.99986 1.00000
\end{lstlisting}

\textcolor{red}{a) Vemos que as duas primeira componentes já conseguem explicar mais de $99.5\%$ da variância, portanto eu diria que duas componentes principais já são suficientes para explicar os dados.}


\begin{lstlisting}
    par(mfrow=c(2,1))
    barplot(pca$rotation[,1]) # Brilho
    barplot(pca$rotation[,2]) # Cor
\end{lstlisting}

\textcolor{red}{b) Para interpretar, decidi fazer um gráfico de barras das componentes e percebi que a primeira componente está relacionada simplesmente com a magnitude das estrelas, ou seja, o brilho aparente delas. Já para a segunda componente, notei que as banda U e I estão com maior peso, logo imagino que essa componente está buscando explicar as cores das estrelas, ou seja, suas temperaturas efetivas, pois se a T$_{eff}$ é baixa, então emitirá mais em comprimentos de onda maiores, enquanto que para T$_{eff}$ alta, temos maior emissão em comprimentos de onda menores.}


\textcolor{red}{c) Para computar as magnitudes nesse espaço, basta utilizar os coeficientes, ou seja, aplicar a transposta da matriz no vetor do espaço do PCA.}

\begin{lstlisting}
    pca_vector = c(0.2, 0.43, 0.6, 0.8, 0.4)
    mags = c()
    for (i in c(1:5)){
      mag = sum(pca$rotation[i,] * pca_vector) + mean(new_df[,i])
      mags = c(mags,mag)
    }
    cat('Magnitudes (U,B,V,R,I): ',mags)

    ### output ###
    Magnitudes (U,B,V,R,I):  19.74012 17.93687 17.82269 16.01448 16.04089
\end{lstlisting}

