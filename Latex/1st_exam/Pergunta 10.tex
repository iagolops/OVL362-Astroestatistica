{$\space$\par}
\vspace{0.5cm}
\justifying
\section*{{\bfseries \LARGE Questão 10 -} {\bfseries \large  As distribuições de z e Vmag de cada classe (Type) podem ser consideradas gaussianas? Justifique sua resposta com mais de um teste/análise.}}

\vspace{0.8cm}


\begin{lstlisting}
    for (tip in c('QSO', 'SY1', 'BLZ')) {
      subset = data[data$Type == tip, ]
      z = na.omit(subset$z)
      v = na.omit(subset$Vmag)
      cat('Starting', tip, 'type\n\n')
    
    # Redshift
      if (length(z) > 7) {
      ad_z = ad.test(z)
      cvm_z = cvm.test(z)
      cat('p-value from AD test for redshift in',tip,':',ad_z$p.value,'\n')
      cat('p-value from CVM test for redshift in',tip,':',cvm_z$p.value,'\n')
      }
    
      dip = dip.test(z)
      shapiro = shapiro.test(z)
      lillie = lillie.test(z)
      cat('p-value from dip test for redshift in',tip,':',dip$p.value,'\n')
      cat('p-value from Shapiro test for redshift in',tip,':',shapiro$p.value,'\n')
      cat('p-value from Lillie test for redshift in',tip,':',lillie$p.value,'\n','\n')
    
    # V mag
    
      if (length(v) > 7) {
        ad_v = ad.test(v)
        cvm_v = cvm.test(v)
        cat('p-value from AD test for Vmag in', tip, ':', ad_v$p.value, '\n')
        cat('p-value from CVM test for Vmag in', tip, ':', cvm_v$p.value, '\n')
      }
    
      dip = dip.test(v)
      shapiro = shapiro.test(v)
      lillie = lillie.test(v)
      cat('p-value from dip test for Vmag in', tip, ':', dip$p.value, '\n')
      cat('p-value from Shapiro test for Vmag in', tip, ':', shapiro$p.value, '\n')
      cat('p-value from Lillie test for Vmag in', tip, ':', lillie$p.value, '\n', '\n')
      cat('###########################################################\n\n')
    }
\end{lstlisting}

\newpage


\begin{lstlisting}
    Starting QSO type
    
    p-value from AD test for redshift in QSO : 7.681129e-09 
    p-value from CVM test for redshift in QSO : 1.121454e-07 
    p-value from dip test for redshift in QSO : 0.9870413 
    p-value from Shapiro test for redshift in QSO : 1.547669e-06 
    p-value from Lillie test for redshift in QSO : 1.573389e-05 
     
    p-value from AD test for Vmag in QSO : 0.03655097 
    p-value from CVM test for Vmag in QSO : 0.05203636 
    p-value from dip test for Vmag in QSO : 0.4685278 
    p-value from Shapiro test for Vmag in QSO : 0.03447447 
    p-value from Lillie test for Vmag in QSO : 0.06412205 
     
    ###########################################################
    
    Starting SY1 type
    

    p-value from dip test for redshift in SY1 : 0.1182018 
    p-value from Shapiro test for redshift in SY1 : 0.4400546 
    p-value from Lillie test for redshift in SY1 : 0.6378789 

    p-value from dip test for Vmag in SY1 : 0.9351645 
    p-value from Shapiro test for Vmag in SY1 : 0.2234065 
    p-value from Lillie test for Vmag in SY1 : 0.2478134 
     
    ###########################################################
    
    Starting BLZ type
    
    p-value from AD test for redshift in BLZ : 0.006366748 
    p-value from CVM test for redshift in BLZ : 0.01736082 
    p-value from dip test for redshift in BLZ : 0.7571165 
    p-value from Shapiro test for redshift in BLZ : 0.003163003 
    p-value from Lillie test for redshift in BLZ : 0.08508485 
     
    p-value from AD test for Vmag in BLZ : 0.06981848 
    p-value from CVM test for Vmag in BLZ : 0.06846265 
    p-value from dip test for Vmag in BLZ : 0.9826913 
    p-value from Shapiro test for Vmag in BLZ : 0.1373625 
    p-value from Lillie test for Vmag in BLZ : 0.119869 
     
    ###########################################################
\end{lstlisting}

\vspace{0.4cm}

\textcolor{red}{Da análise de quasares, pode-se dizer que temos evidência para confirmar que as distribuições de redshift e de magnitude V \textbf{não} seguem uma normal. Apesar do p valor alto em alguns testes, como o dip, estou considerando as diversas características de uma distribuição normal, por isso faço tal afirmação.}

\textcolor{red}{Da análise de Seyfert 1, pode-se dizer que não há evidência para confirmar a refutar a hipótese nula de que as distribuições redshift e da magnitude V seguem uma normal.}

\textcolor{red}{Da análise de blazares, pode-se dizer que temos evidência para confirmar que as distribuições de redshift não segue uma normal. Para a magnitude V, não temos evidência para refutar a hipótese nula.}