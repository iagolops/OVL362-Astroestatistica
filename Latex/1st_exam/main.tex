\documentclass[oneside]{book}
\usepackage[utf8]{inputenc}
\usepackage{fancyhdr}
\usepackage[spanish]{babel}
\usepackage[document]{ragged2e}
\usepackage{lastpage}
\usepackage{amsmath}
\usepackage{listings}
\usepackage{xcolor}
\usepackage{enumitem}
\usepackage{siunitx}
\usepackage{amssymb}
\usepackage{mathrsfs, amsmath}
\usepackage{amsfonts}
\usepackage{amsthm}
\usepackage{tcolorbox}
\usepackage{cancel}
\usepackage{siunitx}
\usepackage{nccmath}
\usepackage{hyperref,lipsum}
\usepackage{ragged2e}
\usepackage{fullpage}

\lstset{
    language=R,
    basicstyle=\ttfamily\small,
    keywordstyle=\color{blue},
    stringstyle=\color{red},
    commentstyle=\color{gray},
    showstringspaces=false,
    breaklines=true,
    backgroundcolor=\color{lightgray}, % <<< fundo cinza
    frame=single,                      % <<< borda ao redor
    literate={_}{{\_}}1                % <<< tratar "_" normalmente
}




\setlength{\headheight}{12.0pt} % Set the headheight to at least 12.0pt
\addtolength{\topmargin}{-12.0pt} % Adjust topmargin to compensate

\pagestyle{fancy}
\fancyhf{}
\newcommand{\Lagr}{\mathcal{L}}
\lhead[Head 1]{P1 de Astroestatística}
\rhead[Head 2]{Iago Lopes}
\rfoot[Página \thepage]{Página \thepage}
\begin{document}

\begin{titlepage}
\centering
{\bfseries\LARGE Universidade Federal do Rio de Janeiro \par}
\vspace{0.1cm}
{\bfseries\LARGE Observatório do Valongo \par}
\vfill
\noindent\hrulefill \\
{\scshape\Huge Segunda prova\par} %Título
\vspace{0.5cm}
{\scshape\Large Astroestatística \par}
\vspace{0.5cm}
{\scshape\Large 2025.1 \par} %Semestre
\noindent\hrulefill \\
\vfill
{\bfseries\Large Iago Lopes\\ \small DRE: 122077032 \par}
\vfill
{\Large Professor: Hélio Jaques \par}
\vfill
{\large 17 de Julho 2025 \par} %esto crea la fecha de hoy
\end{titlepage}
\setcounter{page}{1}
\newpage
{$\space$\par}
\vspace{0.5cm}
\justifying
\section*{{\bfseries \LARGE Questão 1 -} {\bfseries \large  Represente cada aluno por uma Face de Chernoff. A partir delas, você julga que há alunos que apresentem um histórico no ENEM e CRA do primeiro período similar entre si? Ou não?}}

\vspace{0.2cm}

\textcolor{red}{Um bom método para exploração e classificação é o de Faces de Chernoff que utiliza característica físicas para exemplificar o quão similar os dados são no espaço de seus parâmetros. Duas faces similares indicam pontos que estão em uma região próxima no espaço dos parâmetros. Esse método facilita para humanos avaliarem similaridades nos dados e então poder trabalhar em cima dessa análise inicial. Para isso usei a função \texttt{faces} considerando todas as colunas da amostra de alunos da astronomia, exceto seus nomes e o ano da turma, já que não são relevantes ao olhar para a nota do Enem e o CRA.}

\vspace{0.2cm}

\begin{lstlisting}
    install.packages('TeachingDemos')
    library(TeachingDemos)

    astro = read.table('/content/ENEM_Astro.csv', sep=',', header=T)
    
    options(repr.plot.width=14,repr.plot.height=12)
    # Usando todas colunas, exceto o nome para criar as faces
    faces(astro[, 2:7], labels = astro$Nome)
\end{lstlisting}


\begin{figure}[h]
    \centering
    \includegraphics[width=0.8\linewidth]{Figuras/Faces.png}
    \caption{Resultado ao aplicar o método de faces de Chernoff na amostra de alunos da astronomia.}
    \label{faces}
\end{figure}

\textcolor{red}{O resultado pode ser visto na imagem \ref{faces}, onde é possível dizer que de fato há alunos similares nesse espaço de parâmetros. Os principais grupo que percebi foi o de: (S1, B), (S2, M3) e (L1, L2, L3) como indicado na imagem.}
\newpage
{$\space$\par}
\vspace{0.5cm}
\justifying
\section*{{\bfseries \LARGE Questão 2 -} {\bfseries \large Com base nessas notas, você diria que o CRA do primeiro período é uma função linearmente proporcional ao desempenho do aluno no ENEM?}}

\vspace{0.2cm}

\textcolor{red}{Para analisar rigorosamente, irei fazer um ajuste linear simples e obter o valor p. A minha métrica de desempenho para o ENEM será apenas a média ponderada, onde assumirei peso igual para todas matérias.}


\begin{lstlisting}
    install.packages('MASS')
    library(MASS)
    
    # Data
    enem_mean = rowMeans(astro[,c(2:6)])
    cra = astro$CRA
    
    # Reg
    reg = lm(cra ~ enem_mean,method='MM')
    summary(reg)
    
    ### output ### 
    Call:
    lm(formula = cra ~ enem_mean, method = "MM")
    
    Residuals:
        Min      1Q  Median      3Q     Max 
    -3.0514 -0.9285  0.1149  1.2654  1.8299 
    
    Coefficients:
                 Estimate Std. Error t value Pr(>|t|)  
    (Intercept) -7.965886   5.982551  -1.332   0.2017  
    enem_mean    0.021012   0.008508   2.470   0.0252 *
    ---
    Signif. codes:  0 ‘***’ 0.001 ‘**’ 0.01 ‘*’ 0.05 ‘.’ 0.1 ‘ ’ 1
    
    Residual standard error: 1.525 on 16 degrees of freedom
    Multiple R-squared:  0.276,	Adjusted R-squared:  0.2308 
    F-statistic:   6.1 on 1 and 16 DF,  p-value: 0.02515
    
    # Plot
    x = c(min(enem_mean):max(enem_mean),100)
    plot(enem_mean, cra, cex=2.5, col='red',pch=16,xlab='Média ENEM',
         ylab='CRA',cex.lab=2.5,cex.axis=2) 
    abline(reg, col='blue', lwd=3)
    legend('topleft',legend=c('Alunos','Ajuste linear'),col=c('red','blue'),
    pch=c(16,NA),lty = c(NA, 1),lwd = c(NA, 3),cex=c(1.5,1.5),bty='n')
    
    # Text
    text(640, 3.5, paste('Correlacao (Spearman):', correlation, "\n p-valor:",
     round(test$p.value,3)),
         cex=2.5,
         font=2)
\end{lstlisting}

\begin{figure}
    \centering
    \includegraphics[width=0.8\linewidth]{Figuras/corr_enem.png}
    \caption{CRA em função da média ponderada no ENEM dos alunos da astronomia das turmas de 2024 e 2025.}
    \label{corr_enem}
\end{figure}


\textcolor{red}{Ao observar o valor p do modelo linear na figura \ref{corr_enem}, não podemos fazer nenhuma afirmação sobre a linearidade entre o CRA e o desempenho no ENEM.}
\newpage
{$\space$\par}
\vspace{0.5cm}
\justifying
\section*{{\bfseries \LARGE Questão 3 -} {\bfseries \large  Um determinado modelo de formação estelar prevê que cerca de 38\% das estrelas de um aglomerado globular devem ser binárias. Seu colaborador estudou 568 estrelas de um aglomerado globular e encontrou que 200 delas devem ser binárias. Com base nesses dados, você pode refutar o modelo mencionado acima?}}

\vspace{0.8cm}

\textcolor{red}{Para isso, precisamos utilizar um teste de hipótese baseado em proporções, que é o mais adequado ao problema:}

\vspace{0.4cm}

\begin{lstlisting}
    prop.test(200, n = 568, p=0.38)
\end{lstlisting}

\begin{lstlisting}
    	1-sample proportions test with continuity correction
    
    data:  200 out of 568, null probability 0.38
    X-squared = 1.7584, df = 1, p-value = 0.1848
    alternative hypothesis: true p is not equal to 0.38
    95 percent confidence interval:
     0.3130942 0.3931624
    sample estimates:
            p 
    0.3521127 
\end{lstlisting}

\vspace{0.4cm}

\textcolor{red}{De acordo com o teste de proporções, não podemos refutar o modelo mencionado, pois seu p-valor é de $0.38$ e nessa análise estamos considerando como evidência forte valores p menores do que $0.05$.}
\newpage
{$\space$\par}
\vspace{0.5cm}
\justifying
\section*{{\bfseries \LARGE Questão 4 -} {\bfseries \large  Crie uma função no R que receba um vetor de dados qualquer e forneça como resultado ao usuário: na tela gráfica, uma figura de dois painéis, um ao lado do outro, nos quais o primeiro mostra um gráfico qq-plot para a amostra fornecida e uma linha qqline vermelha tracejada de uma população normal; e no segundo, um histograma do vetor fornecido, acompanhado por um “tapete" de valores individuais na abscissa (comando rug) e de três barras verticais, sendo a central em linha sólida vermelha indicando o valor da média amostral, e as duas linhas laterais tracejadas verdes indicando o erro da média amostral, isto é, $\mu\pm \epsilon\mu$. Lembre-se que o erro da média amostral costuma ser chamado de “erro padrão”. Ainda, adicione nesse último painel a curva da melhor gaussiana ajustada aos dados fornecidos.}}

\vspace{0.2cm}

\textcolor{red}{O plot de quantil-quantil é muito útil para uma análise exploratória da gaussianidade dos dados. Além dele, também podemos fazer uma investigação visual no histograma dos dados e comparar com um ajuste da melhor gaussiana. Para essa análise, utilizando o método de máxima verossimilhança, que seleciona os parâmetros que maximizam $\prod_{i=1}^nf(X_i;\theta)$ no ajuste.
Para demostrar a função, criei um vetor com a soma de 3 distribuições diferentes e os resultados se encontram na imagem.
}

\vspace{0.8cm}

\begin{figure}[h]
    \centering
    \includegraphics[width=0.9\linewidth]{Figuras/Captura de tela 2025-06-01 105629.png}
\end{figure}

\newpage

\begin{lstlisting}
    analysis = function(x){
        par(mfcol=c(1,2))
        options(repr.plot.width=18,repr.plot.height=10)
        
        # QQ plot
        qqnorm(x, xlab='Gausian quantiles', ylab = 'Data quantiles', main='QQ plot')
        qqline(x, col='red', lwd=3, lty=2)
        
        # Histogram
        hist(x, breaks=seq(min(x), max(x), length.out = 100))
        rug(x)
        
        # Estimating mean and its error
        results = t.test(x)
        abline(v=results$estimate, col='red', lwd=3)
        abline(v=results$conf.int[1], col='green', lty=2, lwd=3)
        abline(v=results$conf.int[2], col='green', lty=2, lwd=3)
        
        # Fitting a gaussian dist
        fit = fitdist(x, 'norm', 'mle')
        x_plot = seq(min(x), max(x), length.out = 100)
        weight = (max(x)-min(x))/100 * length(x)
        y_plot = dnorm(x_plot, mean=fit$estimate[1], sd=fit$estimate[2])*weight
        lines(x_plot, y_plot, lwd=4)
        legend('topleft', legend='Gaussian fit', lwd=4, col='black', box.lwd=0)
    }
    
    vetor = rpois(1000,lambda = 3)+rnorm(1000) + 6*rchisq(1000,df=4)
    analysis(vetor)
\end{lstlisting}


\newpage
{$\space$\par}
\vspace{0.5cm}
\justifying
\section*{\bfseries \LARGE Questão 5 - \large A galáxia NGC 5548 hospeda um AGN classificado como Seyfert 1. Peterson et al. (1999) observaram seu fluxo durante 8 anos no contínuo óptico (em 5100 Å) e na linha H$\beta$. O arquivo \texttt{NGC5548.dat} contém a data juliana da observação (JD), os fluxos no contínuo óptico e na linha H$\beta$ (F5100 e FHbeta), bem como seus respectivos erros (e\_F5100 e e\_FHbeta). Os fluxos estão medidos em erg cm$^{-2}$ s$^{-1}$ Å$^{-1}$.}

\vspace{0.3cm}

\begin{enumerate}
    \item Suponha que seja possível descrever o fluxo no contínuo óptico deste AGN a partir do fluxo na linha H$\beta$, e que apenas essas duas informações sejam conhecidas. Que parametrização seria essa?
        
    \item Suponha agora que as medidas de erro sejam conhecidas. Critique o resultado obtido acima com base no que esses erros indicam. Represente ambas as distribuições empíricas em um gráfico quantil-quantil.

    \item Conhecendo os valores medidos para os fluxos e seus erros estimados, sugira uma parametrização mais adequada. Explique por que ela é mais adequada a esse problema e como ela foi obtida. 
\end{enumerate}
\vspace{0.8cm}

\textcolor{red}{A) Essa parametrização seria uma regressão linear ordinária, onde não leva-se em conta possíveis erros nas variáveis dependente e independente. Para isso, usei o linear model (lm) do R.}

\vspace{0.8cm}

\begin{lstlisting}
    ngc = read.table('/content/NGC5548.dat', sep='|', header = T)
    ngc = ngc[complete.cases(ngc),]
    
    options(repr.plot.width=10,repr.plot.height=10)
    
    fit = lm(F5100 ~ FHbeta, data = ngc)
    
    x_vals = seq(min(ngc$FHbeta), max(ngc$FHbeta), length.out = 200)
    y_pred = predict(fit, newdata = data.frame(FHbeta = x_vals))
    
    plot(ngc$FHbeta, ngc$F5100, pch=20, xlab="FHbeta", ylab="F5100",
    xlim=c(6,14), ylim=c(7,15))
    lines(x_vals, y_pred, col = "blue", lwd = 2)
    
    for (i in 1:length(ngc$FHbeta)) {
      lines(
        c(ngc$FHbeta[i], ngc$FHbeta[i]),
        c(ngc$F5100[i] + ngc$e_F5100[i], ngc$F5100[i] - ngc$e_F5100[i]),
        col=gray(0.5)
        )
      lines(
        c(ngc$FHbeta[i] + ngc$e_FHbeta[i], ngc$FHbeta[i] - ngc$e_FHbeta[i]),
        c(ngc$F5100[i], ngc$F5100[i]),
        col=gray(0.5)
        )
    }
\end{lstlisting}


\begin{figure}[h]
    \centering
    \includegraphics[width=0.6\linewidth]{Figuras/fit_a.png}
    \caption{Regressão linear ordinária do fluxo contínuo no ótico em relação ao fluxo na linha H$\beta$ em azul. Os pontos mostram os dados com as barras de erro.}
    \label{fit_a}
\end{figure}


\textcolor{red}{B) Se as medidas de erro são conhecidas, então devemos utilizar um método que leve em conta possíveis erro nas medidas. Além disso, deve-se usar a variável com menor erro como variável independente e para verificar qual tem menor erro, eu calculei o RMS do erro relativo de ambas quantidades. Logo, deve-se utilizar o fluxo em contínuo no ótico como variável independente.}

\begin{lstlisting}
    rms_rel_F5100 = sqrt(mean((ngc$e_F5100 / ngc$F5100)^2))
    rms_rel_FHbeta = sqrt(mean((ngc$e_FHbeta / ngc$FHbeta)^2))
    
    cat("Erro RMS relativo em F5100:", round(rms_rel_F5100, 3), "\n")
    cat("Erro RMS relativo em FHbeta:", round(rms_rel_FHbeta, 3), "\n")

    ### output ###
    Erro RMS relativo em F5100: 0.033 
    Erro RMS relativo em FHbeta: 0.034 
\end{lstlisting}

\textcolor{red}{C) A parametrização mais adequada para esse caso é o método de mínimos quadrados pesado pelo erro e utilizando o valor com menor erro como variável independente. Para isso, usei o linear model (lm) usando os erros como peso.}

\begin{lstlisting}
    weight = 1 / (ngc$e_FHbeta * ngc$e_FHbeta)
    
    fit = lm(FHbeta ~ F5100, data = ngc, weights = weight)
    
    x_vals = seq(min(ngc$F5100), max(ngc$F5100), length.out = 200)
    y_pred = predict(fit, newdata = data.frame(F5100 = x_vals))
    
    plot(ngc$F5100, ngc$FHbeta, pch = 20, ylab = "FHbeta", xlab = "F5100",
    ylim=c(6,14), xlim=c(7,15))
    lines(x_vals, y_pred, col = "blue", lwd = 2)
    
    for (i in 1:nrow(ngc)) {
      lines(
        c(ngc$F5100[i] - ngc$e_F5100[i], ngc$F5100[i] + ngc$e_F5100[i]),
        c(ngc$FHbeta[i], ngc$FHbeta[i]),
        col = gray(0.5)
      )
      
      lines(
        c(ngc$F5100[i], ngc$F5100[i]),
        c(ngc$FHbeta[i] - ngc$e_FHbeta[i], ngc$FHbeta[i] + ngc$e_FHbeta[i]),
        col = gray(0.5)
      )
    }
\end{lstlisting}

\begin{figure}[h]
    \centering
    \includegraphics[width=0.6\linewidth]{Figuras/fit_b.png}
    \caption{Regressão linear pesada pelo erro do fluxo na linha H$\beta$ em relação ao fluxo contínuo no ótico em azul. Os pontos mostram os dados com as barras de erro.}
    \label{fit_b}
\end{figure}
\newpage
{$\space$\par}
\vspace{0.5cm}
\justifying
\section*{{\bfseries \LARGE Questão 6 -} {\bfseries \large Transforme as cores em magnitudes U, B, R e I, e junte essas magnitudes à V em uma nova dataframe. Aplique a técnica de componentes principais nesta dataframe formada apenas por magnitudes aparentes.}}

\vspace{0.3cm}

\begin{enumerate}
    \item Quantas componentes principais poderiam explicar apropriadamente a distribuição das estrelas nesse espaço de magnitudes?
        
    \item Interprete. O que deve significar fisicamente os dois primeiros componentes principais?

    \item Quais seriam as magnitudes de uma estrela hipotética que pudesse ser representada no espaço de componentes principais pelas coordenadas (0.2, 0.43, 0.6, 0.8, 0.4)?
\end{enumerate}
\vspace{0.8cm}

\textcolor{red}{O método de PCA rotaciona o espaço dos parâmetros de maneira que os primeiros eixos sejam aqueles onde há maior variância dos dados. Esse método é muito útil para reduzir a dimensionalidade do problema ao assumir que as componentes de menor variância são ruídos.}

\vspace{0.4cm}

\begin{lstlisting}
    catalog = read.table('/content/king5.tsv', sep='|', header=T)

    # Estimating magnitude
    Bmag = catalog$BV+catalog$Vmag
    Umag = catalog$UB+Bmag
    Rmag = -(catalog$VR-catalog$Vmag)
    Imag = -(catalog$VI-catalog$Vmag)
    
    new_df = data.frame(Umag = Umag, Bmag = Bmag, Vmag = catalog$Vmag, Rmag = Rmag,
     Imag = Imag
    )
    new_df = new_df[complete.cases(new_df),]
    
    # PCA
    pca = prcomp(new_df)
    summary(pca)
    
    ### output ### 
    Importance of components:
                              PC1     PC2     PC3     PC4     PC5
    Standard deviation     2.7876 0.42396 0.17145 0.04079 0.03337
    Proportion of Variance 0.9735 0.02252 0.00368 0.00021 0.00014
    Cumulative Proportion  0.9735 0.99597 0.99965 0.99986 1.00000
\end{lstlisting}

\textcolor{red}{a) Vemos que as duas primeira componentes já conseguem explicar mais de $99.5\%$ da variância, portanto eu diria que duas componentes principais já são suficientes para explicar os dados.}


\begin{lstlisting}
    par(mfrow=c(2,1))
    barplot(pca$rotation[,1]) # Brilho
    barplot(pca$rotation[,2]) # Cor
\end{lstlisting}

\textcolor{red}{b) Para interpretar, decidi fazer um gráfico de barras das componentes e percebi que a primeira componente está relacionada simplesmente com a magnitude das estrelas, ou seja, o brilho aparente delas. Já para a segunda componente, notei que as banda U e I estão com maior peso, logo imagino que essa componente está buscando explicar as cores das estrelas, ou seja, suas temperaturas efetivas, pois se a T$_{eff}$ é baixa, então emitirá mais em comprimentos de onda maiores, enquanto que para T$_{eff}$ alta, temos maior emissão em comprimentos de onda menores.}


\textcolor{red}{c) Para computar as magnitudes nesse espaço, basta utilizar os coeficientes, ou seja, aplicar a transposta da matriz no vetor do espaço do PCA.}

\begin{lstlisting}
    pca_vector = c(0.2, 0.43, 0.6, 0.8, 0.4)
    mags = c()
    for (i in c(1:5)){
      mag = sum(pca$rotation[i,] * pca_vector) + mean(new_df[,i])
      mags = c(mags,mag)
    }
    cat('Magnitudes (U,B,V,R,I): ',mags)

    ### output ###
    Magnitudes (U,B,V,R,I):  19.74012 17.93687 17.82269 16.01448 16.04089
\end{lstlisting}


\newpage
{$\space$\par}
\vspace{0.5cm}
\justifying
\section*{{\bfseries \LARGE Questão 7 -} {\bfseries \large Aplique uma decomposição de misturas às magnitudes U, B, V, R e I do aglomerado King 5 considerando que cada componente seja uma normal multivariacional.
}}

\vspace{0.3cm}

\begin{enumerate}
    \item Quantas componentes normais multivariacionais são necessárias para explicar a distribuição dessas magnitudes?
        
    \item Quais são as magnitudes médias e a matriz de covariância de cada uma dessas componentes?

    \item Faça dois diagramas cor—magnitude, um ao lado do outro. O primeiro deve ser B−V × V; o segundo deve ser R−I × R. Use o argumento col do comando plot para identificar cada objeto com base na classificação atribuída a ele pela decomposição de misturas.
\end{enumerate}
\vspace{0.8cm}

\textcolor{red}{a) De acordo com o modelo de decomposição de misturas, temos 3 componentes necessárias para explicar a distribuição.}

\begin{lstlisting}
    install.packages('mclust')
    library(mclust)
    model = Mclust(new_df,modelNames = 'VVV')
    summary(model)
    
    ### output ###
    ---------------------------------------------------- 
    Gaussian finite mixture model fitted by EM algorithm 
    ---------------------------------------------------- 
    
    Mclust VVV (ellipsoidal, varying volume, shape, and orientation) model with 3
    components: 
    
     log-likelihood   n df      BIC      ICL
           618.4788 189 62 911.9692 886.6007
    
    Clustering table:
     1  2  3 
    37 85 67 
\end{lstlisting}

\vspace{2em}

\textcolor{red}{b) O vetor das médias pode ser visto no output do código, junto com as matrizes de covariância. Esses valores descrevem completamente as 3 gaussianas multivariacionais encontradas pelo mclust.}

\begin{lstlisting}
    for (i in c(1:3)){
      cat('\nMédia da componente',i,':',model$parameters$mean[,i])
    }
    for (i in 1:3) {
      cat('Matriz componente', i, ':\n')
      print(model$parameters$variance$sigma[,,i])
      cat('\n')
    }
    
    ### output ###
    Média da componente 1 : 19.08092 18.2044 16.9453 16.21884 15.41783
    Média da componente 2 : 18.29859 17.73985 16.61541 15.98515 15.26787
    Média da componente 3 : 20.14439 19.56443 18.22695 17.45339 16.62603

    Matriz componente 1 :
             Umag     Bmag     Vmag     Rmag     Imag
    Umag 3.645411 3.224176 2.959774 2.766449 2.650813
    Bmag 3.224176 3.386730 3.149322 2.955543 2.866553
    Vmag 2.959774 3.149322 2.993255 2.820833 2.780000
    Rmag 2.766449 2.955543 2.820833 2.674180 2.646377
    Imag 2.650813 2.866553 2.780000 2.646377 2.660136
    
    Matriz componente 2 :
              Umag      Bmag      Vmag      Rmag      Imag
    Umag 0.7108262 0.7305238 0.6625979 0.6187826 0.5740213
    Bmag 0.7305238 0.7592262 0.6867012 0.6400591 0.5929057
    Vmag 0.6625979 0.6867012 0.6250097 0.5855402 0.5447514
    Rmag 0.6187826 0.6400591 0.5855402 0.5511798 0.5147319
    Imag 0.5740213 0.5929057 0.5447514 0.5147319 0.4824006
    
    Matriz componente 3 :
              Umag      Bmag      Vmag      Rmag      Imag
    Umag 0.3867627 0.4024006 0.3650820 0.3445183 0.3291485
    Bmag 0.4024006 0.4465300 0.4134285 0.3957304 0.3819135
    Vmag 0.3650820 0.4134285 0.3919677 0.3794756 0.3695648
    Rmag 0.3445183 0.3957304 0.3794756 0.3711844 0.3638454
    Imag 0.3291485 0.3819135 0.3695648 0.3638454 0.3586400
\end{lstlisting}

\vspace{2em}

\begin{figure}[h]
    \centering
    \includegraphics[width=0.8\linewidth]{Figuras/mistura.png}
    \caption{Estrelas do aglomerado aberto King 5 em dois diagramas cor x magnitude. As cores mostram o grupo classificado pela separação de misturas usando o mclust.}
    \label{misturas}
\end{figure}

\textcolor{red}{c) Podemos ver que de fato há 2 populações na imagem \ref{misturas}, onde os objetos do grupo em azul são estrelas mais avermelhadas e com menor brilho aparente, enquanto que o grupo verde são estrelas mais azuladas e maior brilho aparente. Como é um aglomerado e as estrelas devem estar aproximadamente a mesma distância, podemos dizer que esse gráfico se aproxima bastante de um diagrama HR do aglomerado. O terceiro grupo parece ser construídos de outliers, possivelmente objetos fora do aglomerado ou estrelas fora da sequência principal.}

\begin{lstlisting}
    options(repr.plot.width=16,repr.plot.height=8)
    par(mfrow=c(1,2))
    plot(c(), xlim=range(new_df$Vmag), ylim=range(new_df$Bmag - new_df$Vmag),
         xlab="Vmag", ylab="B-V", main="Objetos classificados")
    for (i in c(1:3)){
      sub_df = new_df[model$classification==i,]
      points(sub_df$Vmag,sub_df$Bmag-sub_df$Vmag, col=1+i, pch=14+i,cex=2)
    }
    
    plot(c(), xlim=range(new_df$Vmag), ylim=range(new_df$Bmag - new_df$Vmag),
         xlab="Rmag", ylab="R-I", main="Objetos classificados")
    for (i in c(1:3)){
      sub_df = new_df[model$classification==i,]
      points(sub_df$Rmag,sub_df$Rmag-sub_df$Imag, col=1+i, pch=14+i,cex=2)
    }
\end{lstlisting}



\newpage
{$\space$\par}
\vspace{0.5cm}
\justifying
\section*{{\bfseries \LARGE Questão 8 -} {\bfseries \large Tasse et al. (2011) mediu a emissão de raios x em AGNs para estudar a relação entre a atividade do AGN e a taxa de formação estelar na galáxia hospedeira. Seus dados se encontram no arquivo xrays\_tasse.tsv. As colunas são nome do AGN (Name), log fluxo em raios X moles (logFxs), log fluxo em raios X duros (logFxh), magnitudes gmag, rmag e imag, desvio para o vermelho (zph), logaritmo da massa da galáxia (logM) e logaritmo da taxa de formação estelar (logSFR).}}

\vspace{0.3cm}

\begin{enumerate}
    \item Faça um gráfico de logFxs versus logFxh. Sobreponha a este gráfico as seguintes curvas de regressão, cada uma com cor diferente: OLS, regressão quantílica da mediana, estimador de Nadaraya-Watson e LOESS.
        
    \item Represente a densidade do espaço logM versus logSFR usando um histograma bidimensional e curvas de contorno.
\end{enumerate}

\vspace{0.8cm}

\begin{figure}[h]
    \centering
    \includegraphics[width=0.8\linewidth]{Figuras/fits.png}
    \caption{Dados do fluxo de raios-X moles vs duros. As linhas representam diferente tipos de regressões aplicadas nos dados.}
    \label{fig:enter-label}
\end{figure}

\textcolor{red}{Ao realizar uma regressão podemos escolher diferentes tipos de estimadores e sua escolha dependerá da natureza do problema. Neste cenário, vemos que a regressão local e a regressão usando o estimador de Nadaraya-Watson foram mais suscetíveis aos objetos nos extremos das distribuições, podendo gerar problemas de 'overfitting'.}

\vspace{2em}

\begin{lstlisting}
    # Data
    agns = read.table('/content/xrays_tasse.tsv', header=T, sep='|')
    agns = agns[complete.cases(agns),]
    
    # Packages
    install.packages('quantreg')
    library(quantreg)
    install.packages("np")
    library(np)
    
    # Fitting
    fit_ols = lm(logFxs ~ logFxh, data = agns)
    fit_qtl = rq(logFxs ~ logFxh, data = agns, tau = 0.5)
    fit_nw = npreg(logFxs ~ logFxh, data = agns, bws = 0.2)
    fit_loe = loess(logFxs ~ logFxh, data = agns)
    
    # Points
    x = seq(min(agns$logFxh), max(agns$logFxh), length.out = 200)
    x_df = data.frame(logFxh = x)
    
    # Plotting
    options(repr.plot.width=16, repr.plot.height=8)
    plot(agns$logFxh, agns$logFxs, pch=19, col='gray', xlab='logFxh', ylab='logFxs',
     cex.lab=1.8, cex.axis=1.5)
    lines(x, predict(fit_ols, newdata = x_df), col='blue', lwd=3)
    lines(x, predict(fit_qtl, newdata = x_df), col='red', lwd=3)
    lines(x, predict(fit_nw, exdat = x_df), col='darkgreen', lwd=3)
    lines(x, predict(fit_loe, newdata = x_df), col='orange', lwd=3)
    legend('topleft', legend=c('OLS', 'Quantile', 'Nadaraya-Watson', 'Loess'), 
    col=c('blue','red','darkgreen','orange'), lwd=3, cex=1.5, bty='n')
\end{lstlisting}

\begin{figure}[h]
    \centering
    \includegraphics[width=0.8\linewidth]{Figuras/hist_2d.png}
    \caption{curvas de contorno com histograma 2D da massa da galáxia versus formação estelar, ambos em logaritmo.}
    \label{hist_2d}
\end{figure}


\textcolor{red}{b) Para representar a densidade bidimensional usei kernels gaussianos, onde cada ponto receberá uma distribuição gaussiana 2D com certa largura e ao final todas distribuições serão levadas em conta no histograma 2D. Percebe-se que galáxias com uma baixa taxa de formação estelar são geralmente mais massivas. Isso pode ser explicado pela evolução de galáxias, onde as galáxias mais massivas são elípticas vermelhas que já estão velhas e sem formação de estrelas, enquanto as galáxias de menor massa possuem maior formação estelar e são mais jovens. }

\begin{lstlisting}
    kde = kde2d(agns$logSFR, agns$logM, n=500)
    image(kde, xlab='Log SFR', ylab='Log M', cex.lab=2)
    contour(kde,add=T)
\end{lstlisting}


\newpage
{$\space$\par}
\vspace{0.5cm}
\justifying
\section*{{\bfseries \LARGE Questão 9 -} {\bfseries \large O arquivo AsteroidClass.tsv contém os dados da fotometria de asteroides obtidos por Popescu et al. (2018). As colunas representam as cores YJ, JKs, HKs e a classificação espectroscópica do asteroide. Construa uma árvore de classificação para essa amostra, com base nas cores dos asteroides. 
}}

\vspace{0.8cm}

\textcolor{red}{Para criar a árvore de classificação, utilizei as cores para explicar a classe espectroscópica. Então analisei o resultado através do sumário e do gráfico jitter \ref{jitter}. Percebe-se diversas classificações erradas, principalmente para objetos que realmente são da classe C. Possivelmente há poucos objetos dessa classe e estão em regiões de sobreposição no espaço multivariacional do problema. Para uma melhor performance, poderíamos usar florestas aleatórias, que é uma extensão de árvores de classificação.}

\begin{figure}[h]
    \centering
    \includegraphics[width=0.7\linewidth]{Figuras/tree.png}
    \caption{Árvore de classificação para a amostra de asteroides.}
    \label{tree}
\end{figure}

\begin{figure}[h]
    \centering
    \includegraphics[width=0.8\linewidth]{Figuras/jitter.png}
    \caption{Gráfico jitter para o resultados da árvore de classificação aplicada na amostra de asteroides.}
    \label{jitter}
\end{figure}

\begin{lstlisting}
    # Data
    asteroid = read.table('/content/AsteroidClass.tsv', header=T, sep='|')
    asteroid = asteroid[complete.cases(asteroid),]

    # Packages
    install.packages('tree')
    library(tree)

    # Creating tree
    arv = tree(as.factor(Class) ~ HKs + JKs + YJ , data=asteroid)
    summary(arv)
    
    ### output ###
    
    Classification tree:
    tree(formula = as.factor(Class) ~ HKs + JKs + YJ, data = asteroid)
    Number of terminal nodes:  9 
    Residual mean deviance:  0.5786 = 4007 / 6926 
    Misclassification error rate: 0.1256 = 871 / 6935 
    
    # Plot
    plot(arv)
    text(arv)
    
    # Jitter plot
    pred = predict(arv, type = "class")
    asteroid$Class = as.factor(asteroid$Class)
    
    plot(jitter(as.numeric(pred), factor=0.5),
         jitter(as.numeric(asteroid$Class), factor=0.5),
         pch=20, cex=0.6,
         xlab='Classe prevista', ylab='Classe verdadeira',
         xlim=c(0.5, 3.5),
         ylim=c(0.5, 3.5),
         axes=FALSE)
    
    axis(1, at=1:3, labels=levels(asteroid$Class))
    axis(2, at=1:3, labels=levels(asteroid$Class))
    box()
\end{lstlisting}

\newpage
{$\space$\par}
\vspace{0.5cm}
\justifying
\section*{{\bfseries \LARGE Questão 10 -} {\bfseries \large  Gaspar et al. (2003) publicaram dados de fotometria de estrelas do aglomerado NGC 2126. Esses dados estão no arquivo ngc2126.dat. Nesta análise, use apenas as componentes de movimento próprio (pmRA e pmDE). Nem todas as estrelas dessa tabela são membros reais de NGC 2126; a rigor, a maioria não deve ser. Para a análise, considere que toda estrela com movimento próprio total nulo provavelmente são estrelas mais distantes e não pertencem ao aglomerado. Entre as estrelas restantes, ainda deve haver algumas intrusas que se encontrem entre nós e o aglomerado; contudo, como o aglomerado se move de forma coesa, ele possui um valor bem marcado em pmRA e pmDE, e as estrelas intrusas serão outliers na distribuição dessas componentes. Use um método de densidade por kernel bidimensional para representar a densidade das estrelas no espaço de componentes do movimento próprio. Estime o movimento próprio mais provável desse aglomerado a partir das coordenadas (pmRA, pmDE) de maior densidade no seu gráfico. Anote-as no gráfico com o símbolo + vermelho, em tamanho cex = 2.5. }}

\vspace{0.8cm}

\begin{figure}[h]
    \centering
    \includegraphics[width=0.8\linewidth]{Figuras/pmda_pmdec.png}
    \caption{Curvas de contorno juntas ao histograma 2D por kenerls das componentes RA e DEC do movimento próprio das estrelas do algomerado. A cruz vermelha indica o valor mais provável para o movimento próprio do aglomerado após separação de misturas.}
    \label{pmda_pmdec}
\end{figure}

\textcolor{red}{Para estimar a densidade bidimensional por kernel, utilizei o MASS:kde2d e o contour no kde para adicionar contornos. Como essa amostra contém outliers que não devem se mover da mesma maneira que o aglomerado, decidi utilizar o \texttt{mclust} para separar as possíveis misturas da amostra. Após realizar a separação em duas componentes gaussianas, o valor mais provável de movimento do aglomerado será o máximo de uma das gaussianas, ou seja, sua média, como mostrado na figura \ref{pmda_pmdec}.} 

\vspace{2em}

\begin{lstlisting}
    # Data
    ngc = read.table('/content/ngc2126.dat', sep='|', header=T)
    mask = sqrt(ngc$pmDE**2+ngc$pmRA**2)>0
    sample = ngc[mask,]

    # Splitting mixture
    model = Mclust(sample[,4:5],modelNames = 'VVV')
    model$parameters$mean

    ### output ###
    A matrix: 2 × 3 of type dbl
    pmRA	-6.517683	-4.151874	-0.9629393
    pmDE	-4.262831	4.439619	-13.9000586

    # Plotting
    kde = kde2d(sample$pmRA, sample$pmDE, n=200)
    image(kde, xlab='pmRA', ylab='pmDEC',col = hcl.colors(100, "terrain"), 
    xlim=c(-20,10),ylim=c(-22,12)) 
    
    #Cmap reference
    #https://www.rdocumentation.org/packages/graphics/versions/3.6.2/topics/image
    contour(kde,add=T)

    # Point
    points(model$parameters$mean[1,1],model$parameters$mean[2,1],pch=3,
    cex=2.5,col='red')
\end{lstlisting}


\newpage



\end{document}