{$\space$\par}
\vspace{0.5cm}
\justifying
\section*{{\bfseries \LARGE Questão 3 -} {\bfseries \large  Um determinado modelo de formação estelar prevê que cerca de 38\% das estrelas de um aglomerado globular devem ser binárias. Seu colaborador estudou 568 estrelas de um aglomerado globular e encontrou que 200 delas devem ser binárias. Com base nesses dados, você pode refutar o modelo mencionado acima?}}

\vspace{0.8cm}

\textcolor{red}{Para isso, precisamos utilizar um teste de hipótese baseado em proporções, que é o mais adequado ao problema:}

\vspace{0.4cm}

\begin{lstlisting}
    prop.test(200, n = 568, p=0.38)
\end{lstlisting}

\begin{lstlisting}
    	1-sample proportions test with continuity correction
    
    data:  200 out of 568, null probability 0.38
    X-squared = 1.7584, df = 1, p-value = 0.1848
    alternative hypothesis: true p is not equal to 0.38
    95 percent confidence interval:
     0.3130942 0.3931624
    sample estimates:
            p 
    0.3521127 
\end{lstlisting}

\vspace{0.4cm}

\textcolor{red}{De acordo com o teste de proporções, não podemos refutar o modelo mencionado, pois seu p-valor é de $0.38$ e nessa análise estamos considerando como evidência forte valores p menores do que $0.05$.}